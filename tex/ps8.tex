\subsection{Planetary belts}
It is generated by the magnetic field of the earth, which direct the particules
to the poles.

\subsection{Rotating disk}
Charge density
\[
	\rho = \frac{Q}{\pi R^2}
\]
Charge on a small circle (small element of charge)
\[
	dq = \rho 2 \pi r dr
\]
Period of rotation
\[
	T = \frac{2 \pi}{w}
\]
Small element of current
\[
	dI = \frac{dq}{T}
	\to dI = \rho w r dr = \frac{Q w}{\pi R^2} r dr
\]
The magnetic field generated from a loop (see course)
\[
	d \vec{B} = \frac{\mu_0 dI}{2 r} \hat{u_z}
	\to d \vec{B} = \frac{\mu_0 Q w}{2 \pi R^2} dr \hat{u_z}
\]
We now integrate the small circle to make the disk
\[
	\vec{B} = \int_{disk} d \vec{B} =
	\to \vec{B} = \int_0^R \frac{\mu_0 Q w}{2 \pi R^2} dr \hat{u_z}
	= \frac{\mu_0 Q w}{2 \pi R^2} \int_0^R dr \hat{u_z}
	= \frac{\mu_0 Q w}{2 \pi R} \hat{u_z}
\]

\subsection{Velocity filter}

See course for same info \\
We want to filter the particules based on speed. Thus, we need to avoid moving
the particule on \(\hat{y}\). We will try to nullifiy it then. We have
\[F_B = q v_x \times B\]
\[F_E = q E\]
\[F_y = F_B + F_E \to F_y = q(E - v_x B)\]
Then, to nullify the force on \(\hat{y}\), we take \(v_x = E / B\)

\subsubsection{a}
\[
	v = \frac{E}{B}
	\to v = \frac{8 \cdot 10^2}{4 \cdot 10^{-3}} = 2 \cdot 10^5
\]

\subsubsection{b}
\[1 \unit{eV} = 1.6 \cdot 10^{-19} \unit{J}\]
We use the kinetic energy
\[E = \frac{1}{2} m v^2\]

\[
	E = \frac{1}{2} 9.109 \cdot 10^{-31} \cdot (2 \cdot 10^5)^2
	= 1.82 \cdot 10^{-20} \unit{J}
	= 1.1375 \cdot 10^{-1} \unit{eV}
	= 113.75 \unit{meV}
\]

\subsubsection{c}

With the deflector, we were able to filter particles with a given charge and
mass ratio. With the given system, we are able to filter particles with a given
speed.

\subsection{The cyclotron}

7\#17: we have a circular movement, so \(a = \frac{v^2}{r}\).
We know that \(F = ma\), and in our case \(F = qvB\) (as \(v \perp B\)),
thus we have that \(qvB = \frac{m v^2}{r} \to qB = \frac{mv}{r}\).

\subsubsection{a}
It makes circle from the center to the outer.

\subsubsection{b}
\[
	qB = \frac{mv}{r}
	\to \frac{v}{r} = w = \frac{qB}{m}
\]
\[
	f = \frac{w}{2 \pi} = \frac{qB}{2 \pi m}
\]

\subsubsection{c BONUS}
We can increase the force of \(\vec{B}\) to compansate the increase of mass.
We can also decrease the frequency.
