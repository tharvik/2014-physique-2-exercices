\subsection{Fields in space}

\subsection{A wire above the ground}
\[
	\int_S \vec{E} \cdot \vec{ds} = \frac{Q}{\epsilon_0}
\]
\[
	V = - \int_{r_b}^{r_a} \vec{E} \vec{dr}
\]
\[
	C = \frac{Q}{V}
\]
\(l\) is the consider length of wire.
\[
	Q = \lambda l
\]
We take a cylinder around the wire.
\[
	\int_S \vec{E} \cdot \vec{ds} = \frac{Q}{\epsilon_0}
	\to E 2 \pi a l = \frac{\lambda l}{\epsilon_0}
	\to E = \frac{\lambda}{\epsilon_0 2 \pi a}
\]
We compute the potential difference with the zero at the ground.
\[
	V = - \int_{r_b}^{r_a} \vec{E} \vec{dr}
	= - \int_{h}^{0} \frac{\lambda}{\epsilon_0 2 \pi a} \vec{dr}
	= - \frac{\lambda}{\epsilon_0 2 \pi a} \int_{h}^{0} \vec{dr}
	= \frac{\lambda h}{\epsilon_0 2 \pi a}
\]
We now have the capacity.
\[
	C = \frac{Q}{V}
	= \frac{l \epsilon_0 2 \pi a}{h}
\]

\subsection{Capacitors}

\[
	C = \frac{Q}{V}
\]
\[
	\sigma = \frac{Q}{S}
\]
\[
	E = \frac{\sigma}{\epsilon_0} = \frac{Q}{\epsilon_0 S}
\]
If there is two fields to consider with different \(S\), \(d\) and \(\epsilon\)
\[
	E_1 = \frac{Q}{\epsilon_{r1} \epsilon_0 S_1}
	\to \Delta V_1 = E_1 d_1 = \frac{Q d_1}{\epsilon_{r1} \epsilon_0 S_1}
	\to C_1 = \frac{\epsilon_{r1} \epsilon_0 S_1}{d_1}
\]
\[
	E_2 = \frac{Q}{\epsilon_{r2} \epsilon_0 S_2}
	\to \Delta V_2 = E_2 d_2 = \frac{Q d_2}{\epsilon_{r2} \epsilon_0 S_2}
	\to C_2 = \frac{\epsilon_{r2} \epsilon_0 S_2}{d_2}
\]
If it is in parallel, we add the two capacities
\[
	C = C_1 + C_2 = \epsilon_0
	( \frac{\epsilon_{r1} S_1}{d_1} + \frac{\epsilon_{r2} S_2}{d_2} )
\]
If it is in serie, we add the two inverses of capacities
\[
	\frac{1}{C} = \frac{1}{C_1} + \frac{1}{C_2}
	= \frac{1}{\epsilon_0}
	( \frac{d_1}{\epsilon_{r1} S_1} +
	\frac{d_2}{\epsilon_{r2} S_2} )
\]

\subsubsection{a}
\[
	d_1 = d
\]
\[
	d_2 = 0
\]
\[
	S_1 = S
\]
\[
	S_2 = 0
\]
\[
	\epsilon_{r1} = 1
\]
\[
	\epsilon_{r2} = 0
\]
\[
	C = C_1 + C_2 = \epsilon_0 \frac{S}{d}
\]
\[
	\frac{1}{C} = \frac{1}{C_1} + \frac{1}{C_2}
	= \frac{d}{S \epsilon_0}
	\to C = \epsilon_0 \frac{S}{d}
\]

\subsubsection{b}
It is in serie
\[
	d_1 = \frac{1}{3} d
\]
\[
	d_2 = \frac{2}{3} d
\]
\[
	S_1 = S_2 = S
\]
\[
	C = \epsilon_0
	( \frac{\epsilon_{r1} S}{\frac{1}{3} d} +
	\frac{\epsilon_{r2} S}{\frac{2}{3} d} )
	= \frac{3 \epsilon_0 S}{d}
	( \epsilon_{r1} + \frac{\epsilon_{r2}}{2} )
	// TODO inverse sides
\]

\subsubsection{c}
It is in parallel
\[
	d_1 = d_2 = d
\]
\[
	S_1 = \frac{1}{3} S
\]
\[
	S_2 = \frac{2}{3} S
\]
\[
	\frac{1}{C} = \frac{1}{C_1} + \frac{1}{C_2}
	= \frac{1}{\epsilon_0}
	( \frac{d}{\epsilon_{r1} \frac{1}{3}} +
	\frac{d}{\epsilon_{r2} \frac{2}{3}} )
	= \frac{3 d}{\epsilon_0}
	( \frac{1}{\epsilon_{r1}} +
	\frac{1}{2 \epsilon_{r2}} )
	= \frac{3 d}{\epsilon_0}
	\frac{2 \epsilon_{r2} + \epsilon_{r1}}{\epsilon_{r1} \epsilon_{r2}}
	\to C = \frac{\epsilon_0}{3 d}
	\frac{\epsilon_{r1} \epsilon_{r2}}{2 \epsilon_{r2} + \epsilon_{r1}}
	// TODO inverse sides
\]

\subsubsection{d}
It is in parallel
\[
	d_1 = d_2 = d
\]
\[
	S_1 = \frac{1}{3} S
\]
\[
	S_2 = \frac{2}{3} S
\]
\[
	\epsilon_{r1} = 1
\]
\[
	\frac{1}{C} = \frac{1}{C_1} + \frac{1}{C_2}
	\to C = \frac{\epsilon_0}{3 d}
	\frac{\epsilon_{r2}}{2 \epsilon_{r2} + 1}
	// TODO inverse sides
\]
