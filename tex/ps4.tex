\subsection{Gauss’s Law}

\subsubsection{a}
As we know that the eletric field generated by a single charge is
\(\frac{q}{\epsilon_0}\). As the cube wrap around the charge and the faces
are at equidistance, we can write
\[
	\phi = \frac{1}{6} \frac{q}{\epsilon_0}
\]

\subsubsection{b}
We can represente this cube as a smaller one inside the one of the previous
question, with is charged corner in the center of the big cube. The faces
containing the charge have a zero eletric field as the flux is \(\parallel\) to
surface
\[
	\phi = \frac{1}{6} \frac{1}{4} \frac{q}{\epsilon_0}
\]

\subsubsection{c}
\[
	\int_S E dS = \frac{Q_{total}}{\epsilon_0} = \frac{\lambda L}{\epsilon_0}
\]
\[
	Q_{total} = \lambda \frac{r^2}{a^2} L
\]
\[
	\int_S E dS = 0 + 0 + E 2 \pi r L
\]
\[
	E = \frac{\lambda}{2 \pi r \epsilon_0} \hat{r}
\]
\[
	\frac{Q_{total}}{\epsilon_0}
	= \frac{\lambda \frac{r^2}{a^2} L}{\epsilon_0}
\]
\[
	E = \frac{\lambda r}{2 \pi \epsilon_0 a^2}
\]

If we take \(r < a\) and \(q = -e\)
\[
	\lambda = n q \pi a^2
\]
\[
	E = - \frac{n e \pi a^2 r}{2 \pi \epsilon_0 a^2}
	= - \frac{n e r}{2 \epsilon_0}
\]
\[
	\vec{F} = q \vec{E} = - e x - \frac{n e r}{2 \epsilon_0}
	= \frac{n e^2 r}{2 \epsilon_0}
\]

\subsubsection{d}

\subsubsection{e}
If we take \(r < R_1\)
\[
	Q = 0 \to \vec{E} = 0
\]
If we take \(R_1 < r < R_2\)
\[
	Q = q_1 \to
	\vec{E} 4 \pi r^2 = \frac{Q}{\epsilon_0} \to
	\vec{E} = \frac{Q}{4 \pi r^2 \epsilon_0}
	\vec{E} = \frac{q_1}{4 \pi r^2 \epsilon_0}
\]
If we take \(R_2 < r\)
\[
	Q = q_1 + q_2 \to
	\vec{E} = \frac{q_1 + q_2}{4 \pi r^2 \epsilon_0}
\]

\subsubsection{f}

\[
	\frac{q_1}{q_2} = -1
\]

\subsection{Hydrogen reloaded}

\subsubsection{}
\[
	F_{rep} = \frac{1}{4 \pi \epsilon_0} \frac{e^2}{(2 a)^2}
	= \frac{1}{16 \pi \epsilon_0} \frac{e^2}{a^2}
\]
\[
	Q_{total} = 2 e \frac{a^3}{r^3}
\]
\[
	\int_S E dS = \frac{Q_{total}}{\epsilon_0}
	= \frac{2 e a^3}{\epsilon_0 r^3}
\]
\[
	\vec{E} = \frac{2 e a}{4 \pi \epsilon_0 r^3} \hat{r}
	\to \vec{F}_{atr} = \frac{2 e^2 a}{4 \pi \epsilon_0 r^3}
\]
\[
	\frac{2 e^2 a}{4 \pi \epsilon_0 r^3}
	= \frac{1}{16 \pi \epsilon_0} \frac{e^2}{a^2}
	\to 8 a^3 = r^3
	\to 2 a = r
	\to a = \frac{r}{2}
\]

\subsection{Corona Discharge}

\subsubsection{a}

\[
	E_{max} \leq 30 [Kv/cm]
\]
\[
	\vec{E} = \frac{Q}{r \pi \epsilon_0 r^2} \leq 30 \cdot 10^3 \cdot 10^2
\]
\[
	Q_{max} = 30 \cdot 10^3 \cdot 10^2 \cdot 4 \pi \cdot 9 \cdot 10^{-12}
	\cdot 9^2 \cdot 10^{-4} \approx 2.7 [\mu C]
\]

\subsubsection{b}

\[
	V = 90 [kV]
\]
\[
	E = \frac{V}{r} \leq 30 \cdot 10^5
\]
\[
	r \geq \frac{90 \cdot 10^3}{30 \cdot 10^5} = 30 [m]
\]

\subsection{Image Charge}

\subsubsection{a}
Make a "square" of charges, touching walls with negative, remaining wall with
positive, each with a distance of \(a\) from touching walls.

\subsubsection{b}

\subsubsection{c}
\[
	\vec{p} = q d
\]
\[
	\vec{E}_r
	= \frac{2 \vec{p} \cos{\theta}}{4 \pi \epsilon_0 (2r)^3} \hat{r}
	= \frac{2 \vec{p}}{4 \pi \epsilon_0 8 r^3}
\]
