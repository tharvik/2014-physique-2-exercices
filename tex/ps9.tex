\subsection{Magnetic dipole moments and the DC motor}

Magnetic dipole
\[
	\vec{m} = \int \vec{L} d\vec{A}
\]

\subsubsection{a}
Left hand rule: it points downwards

\subsubsection{b}

Lorentz force
\[
	\vec{F} = \int_L I d\vec{l} \times \vec{B}
\]
Torque
\[
	\vec{t} = \vec{r} \times \vec{F}
\]
We have \(\vec{I}\) and \(\vec{B}\), thus we just replace
\[
	\vec{t} = \vec{r} \times \vec{F}
	\to \vec{t} = \vec{r} \times \int_L I d\vec{l} \times \vec{B}
\]

--

First and third segments, the current is parallel to the \(\vec{B}\) field, so
there is no force acting on it

Second and fourth segments, the current is perpendicular


\subsubsection{c}
\subsubsection{d}

\(\vec{B}\): If the magnetic field is strong, it will turn faster, if there is
no field, there will be no force

\(\vec{I}\): , A

\subsection{Ampere's Law}

\subsubsection{a}



\subsubsection{b}
\subsubsection{c}

\subsection{Moving Electron Beams}
\subsubsection{a}
\subsubsection{b}
\subsubsection{c}

\subsection{The Hall Effect}
\subsubsection{a}
\subsubsection{b}
